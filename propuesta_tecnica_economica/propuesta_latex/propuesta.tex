\documentclass{article}
\usepackage[utf8]{inputenc}
\usepackage[spanish]{babel}
\usepackage{graphicx}
\usepackage{float}
\usepackage{adjustbox}
\usepackage{fancyhdr}

\pagestyle{fancy}
\fancyhf{}
\lhead{KeyPaX, Barbaro Software}
\rhead{\thepage}
\lfoot{Propuesta técnica y económica}
\renewcommand{\footrulewidth}{0.4pt}

\setlength{\parindent}{2em}
\setlength{\parskip}{1em}

\title{
\Huge{KeyPaX, Barbaro Software} \\
\begin{Large}
Propuesta técnica y económica    del proyecto
\end{Large}}
%\author{}
\date{22 de febrero de 2021}

\begin{document}

\maketitle

\pagebreak

\tableofcontents

\pagebreak

\section{Resumen Ejecutivo}

% Resumen en una página
% Resumen del sistema a desarrollar
% Incluir precio, plazos y entregables

\pagebreak

\section{Objetivos del sistema}

% Descripción breve de las necesidades que el sistema debe abordar y la funcionalidad que se desarrollará para para cubrir estas necesidades.

El sistema KeyPaX permite gestionar una colección de contraseñas de manera segura y remota. Permite almacenar los nombres de usuario y claves de acceso a distintos servicios, además de otra información adicional como URL y descripciones de texto. Estas contraseñas se pueden organizar en categoría y realizar búsquedas según sus campos. El sistema puede generar contraseñas robustas y configurables por el usuario. El usuario accederá a su colección de contraseñas mediante una contraseña maestra y realizará 2FA en las ocasiones requeridas. Se accederá al sistema mediante una interfaz web o una aplicación móvil.

\subsection{Análisis de requisitos preliminar}

% - descripcion de elementos importantes del sistema
% - utilizar requisitos concretos
% - capacidades el sistema
% - "funcionara en android" -> (en requisitos, que version?)
% - no funcionales basicos (que sgbd)
% - ante la duda ponerlo

%Desgranar el punto anterior en forma de requisitos. No tiene que ser el análisis de requisitos totalmente completo y detallado, que se realizará una vez se haya confirmado que el proyecto se va a llevar a cabo, pero es un compromiso con el cliente y debe incluir todo lo importante y estar adecuadamente redactado (con precisión, sin ambigüedades obvias, etc.). Además de texto, se pueden incluir algunos bocetos de la GUI (suelen ser importantes para concretar mejor la funcionalidad de una aplicación).

\begin{table}[H]
    \centering
    \begin{tabular}{| c | p{30em} |}
    \hline
        Código &  Descripción  \\ \hline
        RF-1 & El sistema permite almacenar contraseñas \\ \hline
        RF-1.1 & El sistema permite almacenar pares que constan de nombre de usuario y contraseña  \\ \hline
        RF-1.2 & El sistema permite asociar a las contraseñas una URL del sitio web al que corresponden \\ \hline
        RF-1.3 & El sistema registra la fecha de creación y actualización de la contraseña \\ \hline
        RF-2 & El  sistema permite organizar las contraseñas por categorías \\ \hline
        RF-3 & El sistema permite realizar una búsqueda entre las contraseñas por categoría, fecha de creación y actualización  \\ \hline
        RF-4 & El sistema permite generación de contraseñas pseudoaleatorias \\ \hline
        RF-4.1 & El sistema permite seleccionar la longitud de la contraseña a generar\\ \hline
        RF-4.2 & El sistema permite seleccionar el conjunto de caracteres que compone la contraseña a generar\\ \hline
        RF-4.3 & El sistema mostrará el grado de robustez de la contraseña al ser generada \\ \hline
        RF-5 & El sistema permite al usuario acceder a sus contraseñas únicamente a través de la contraseña maestra \\ \hline
        RF-6 & El sistema requiere \textit{Two-Factor Authentication} (2FA) para iniciar sesión desde un dispositivo nuevo, distinto a los utilizados con anterioridad \\ \hline
        RF-7 & El usuario se registra en el sistema mediante un correo electrónico y una contraseña maestra\\ \hline
        RF-7.1 & El registro de sesión se deberá confirmar, para verificar la identidad, mediante un correo al usuario registrado \\ \hline
        RF-8 & Se accede al sistema mediante una aplicación móvil \\ \hline
        RF-9 & Se accede al sistema mediante una interfaz web \\ \hline
    \end{tabular}
\end{table}


\section{Descripción técnica}

% contraseñas de sistios web cifradas en la base con hash de la master password


\section{Plan de trabajo}
A continuación se va a presentar un primer boceto de plan de trabajo con fechas importantes para los clientes.

\begin{table}[H]
    \centering
    \begin{tabular}{| c | p{30em} |}
    \hline
        Fecha &  Descripción  \\ \hline
        22/02/2021 & Presentación de la propuesta técnica y económica.  \\ \hline
        02/03/2021 & Reunión de feedback con los clientes para aprobación/negociación de la propuesta técnica y económica anteriormente presentada. Comienzo de la redacción del plan de gestión, análisis y diseño. \\ \hline
        15/03/2021 & Presentación de la primera versión del plan de proyecto.\\ \hline
        22/03/2021 & Reunión de seguimiento con los clientes. \\ \hline
        14/04/2021 & Presentación de la segunda versión del plan de proyecto. \\ \hline
        15/04/2021 & Demostración a los clientes de la primera versión desplegada.
        Entrega de documentación generada hasta el momento y transferencia del código de la versión primera del sistema.\\ \hline
        03/05/2021 & Segunda reunión de seguimiento con los clientes. \\ \hline
        21/05/2021 & Demostración a los clientes de la segunda versión desplegada.
        Entrega de documentación generada hasta el momento y transferencia del código de la versión segunda del sistema.\\ \hline
        01/06/2021 & Entrega final del producto. Consistirá en la entrega del código fuente final, manuales de usuario en formato video y texto y delegación de control del producto ya instalado en servidores de acceso públicos.\\ \hline
    \end{tabular}
\end{table}

\section{Equipo técnico encargado del proyecto}

\textbf{Barbarians Software Inc} cuenta con tres años de permanencia en el mercado, que han proporcionado una amplia experiencia en el mundo empresarial y nos avala como una empresa referente en el sector. La empresa se caracteriza por tener una filosofía basada en el trabajo coordinado en grupo, con especialistas en distintos campos del sector que unificados forman un equipo de trabajo altamente sofisticado.
Los servicios y capacidades que nuestros clientes han podido destacar entre los más importantes son:
\begin{itemize}
   \item Experiencia en desarrollo móvil en dispositivos Android.
   \item Veteranía en el FrameWork de JavaScript React para Web y Móvil.
   \item Destreza en el entorno de ejecución NodeJS con el FrameWork Express.
   \item Práctica en Bases de Datos relacionales (Oracle, Postgresql, MySql...).
   \item Conocimiento en Base de Datos no relacionales (MongoDB).
   \item Alta uso de diferentes lenguajes de programación (Java, JavaScript, C, C++, GoLang).
   \item Estudio de metodología de despliegue de Proyectos con Docker y Kubernetes.
   \item Maestría en el uso de IaaS (Microsoft Azure, Amazon AWS).
   \item Prácticas de testing con PostMan.
   \item Altos conocimientos en gestión de versiones a través del Software Git.
\end{itemize}

En cuanto a las características individuales de los 6 integrantes de la empresa:
\begin{itemize}
   \item Arturo Calvera Tonin está actualmente colaborando con el departamento de investigación de la EINA y el I3a en el desarrollo de un sistema de control de acceso a aulas dentro de la universidad.
   \item Jorge Bernal Romero es polivalente en el conocimiento general del proyecto. Altamente implicado en su comunidad local participando frecuentemente en programas de voluntariado.
   \item Andoni Salcedo Navarro tiene un amplio conocimiento y experiencia en la programación con distintos lenguajes en una amplia variedad de ámbitos.
   \item Javier Vela Tambo en vistas a trabajar el año que viene en el extranjero, concretamente en la universidad de Rhode Island en busca de continuar con su formación y ampliar sus conocimientos.
   \item Jorge Borque Benedí está familiarizado con una gran variedad de medios y dispositivos gracias a las distintas experiencias que ha obtenido de distintos dispositivos con los que ha trabajado a lo largo de su trayectoria.
   \item Carlos Bellvis Irache tiene una alta capacidad de liderazgo, motivación de grupo y resolución de conflictos. Esto lo adquiere debido a que actualmente entrena a un equipo de fútbol de categorías inferiores.
\end{itemize}
Estas características individuales permiten afrontar los problemas mediante distintos puntos de vista y con ello conseguir cohesionar en los productos finales un resultado satisfactorio para los clientes.

\section{Presupuesto}

\section*{Anexo I. Glosario}
\addcontentsline{toc}{section}{Anexo I. Glosario}

\section*{Anexo II. Estimación de costes}
\addcontentsline{toc}{section}{Anexo II. Estimación de costes}

%\section*{Anexo III+. Otros anexos}
%\addcontentsline{toc}{section}{Anexo III+. Otros anexos}

\end{document}
