\documentclass{article}
\usepackage[utf8]{inputenc}
\usepackage[spanish]{babel}
\usepackage[hidelinks]{hyperref}
\usepackage{setspace}
\usepackage{graphicx}
\usepackage{float}
\usepackage[hidelinks]{hyperref}

\usepackage[compact]{titlesec}        
\usepackage{geometry}
\geometry{
    a4paper,
    total={170mm,257mm},
    left=30mm, % entre 25-35
    right=30mm, % entre 25-35
    top=30mm, % no más de 30
    bottom=30mm, % no más de 30
}

\usepackage{fancyhdr}
\pagestyle{fancy}
\fancyhf{}
\lhead{}
\rhead{Bárbaros Software S.A.}
\rfoot{\thepage}
\lfoot{Plan de gestión, análisis, diseño y memorial del proyecto KeyPaX}
\renewcommand{\footrulewidth}{0.4pt}

\setlength{\parindent}{2em}
\setlength{\parskip}{1em}

\begin{document}

\begin{titlepage}
    \centering
    \vspace*{\fill}

    \vspace*{0.5cm}

    \Huge 
    \textbf{Plan de gestión, análisis, diseño y memorial del proyecto KeyPaX}
    
    \vspace*{0.5cm}

    \huge
    Bárbaros Software S.A. (Grupo 3)

    \includegraphics[height=5cm]{../images/logo.jpg}

    \begin{Large}
        Carlos Bellvis, 755452\\
        Jorge Bernal, 775695\\
        Jorge Borque, 777959\\
        Arturo Calvera, 776303\\
        Andoni Salcedo, 785649\\
        Javier Vela, 775593\\
    \end{Large}

    \vspace*{1cm}

    \large
    \href{https://github.com/UNIZAR-30226-2021-03}{Organización GitHub: https://github.com/UNIZAR-30226-2021-03}

    \vspace*{\fill}
    
\end{titlepage}

\pagebreak

\tableofcontents

\pagebreak

\section{Introducción}

\section{Organización del proyecto}

El equipo del proyecto está formado por los 6 integrantes del grupo. Para dividir el trabajo y las responsabilidades se han formado 2 grupos iniciales, acorde con las capacidades y conocimientos individuales. Además, se ha designado como \textbf{director del proyecto} a \textbf{Arturo Calvera}, cuya responsabilidad abarca la gestión de los equipos de trabajo y sus responsabilidades. 

\begin{itemize}
    \item \textbf{Equipo \textit{Backend}}:
    \begin{itemize}
        \item Responsable: Arturo Calvera.
        \item Función: Desarrollo del \textit{backend} del sistema.
        \item Integrantes: Arturo Calvera, Andoni Salcedo.
    \end{itemize}
    \item \textbf{Equipo \textit{Frontend}}:
    \begin{itemize}
        \item Responsable: Javier Vela.
        \item Sub-equipos: El equipo dedicado a la vista del sistema se divide para el desarrollo de las dos interfaces. 
        \begin{itemize}
            \item \textbf{Equipo Web}:
            \begin{itemize}
                \item Función: Desarrollo del \textit{frontend} web del sistema.
                \item Integrantes: Jorge Bernal, Javier Vela.
            \end{itemize}
        \item \textbf{Equipo Android}:
            \begin{itemize}
                \item Función: Desarrollo de la aplicación cliente del sistema para dispositivos Android.
                \item Integrantes: Carlos Bellvis, Jorge Borque.
            \end{itemize}
        \end{itemize}
    \end{itemize}
    \item Se contempla la posibilidad de crear nuevos equipos para las necesidades que surjan durante el proyecto (\textit{e.g.} despliegue, pruebas) o la reconfiguración de los actuales para adecuarse a la situación.
    %%% IGUAL ESTO NO MENCIONARLO Y ACTUALIZARLO CUANDO SE CREEN NUEVOS EQUIPOS
\end{itemize}

\section{Plan de gestión del proyecto}

\subsection{Procesos}

\subsubsection{Procesos de inicio}

\subsubsection{Procesos de ejecución y control}

\subsubsection{Procesos técnicos}

\subsection{Planes}

\subsubsection{Plan de gestión de configuraciones}

\subsubsection{Plan de construcción y despliegue del software}

\subsubsection{Plan de aseguramiento de calidad}

\subsubsection{Calendario del proyecto y división del trabajo}

\pagebreak

\section{Análisis y diseño del sistema}

\subsection{Análisis de requisitos}

\begin{table}[H]
    \centering
    \begin{tabular}{| c | p{30em} |}
    \hline
        Código &  Descripción  \\ \hline
        RF-1 & El sistema permite almacenar contraseñas. \\ \hline
        RF-1.1 & El sistema permite almacenar pares que constan de nombre de usuario y contraseña.  \\ \hline
        RF-1.2 & El sistema permite asociar a las contraseñas una URL del sitio web al que corresponden. \\ \hline
        RF-1.3 & El sistema registra la fecha de creación y actualización de la contraseña. \\ \hline
        RF-1.4 & El sistema permite almacenar una descripción de texto asociada a la contraseña. \\ \hline
        RF-1.5 & El sistema permite almacenar ficheros de imagen (jpeg, png, ...) y ficheros PDF, asociados a la contraseña. \\ \hline
        RF-2 & El sistema permite organizar las contraseñas por categorías, fecha de creación y actualización. \\ \hline
        RF-3 & El sistema permite realizar una búsqueda entre las contraseñas por categoría, nombre de usuario.  \\ \hline
        RF-4 & El sistema permite generación de contraseñas pseudoaleatorias. \\ \hline
        RF-4.1 & El sistema permite seleccionar la longitud de la contraseña a generar.\\ \hline
        RF-4.2 & El sistema permite seleccionar el conjunto de caracteres que compone la contraseña a generar.\\ \hline
        RF-4.3 & El sistema mostrará el grado de robustez de la contraseña al ser generada. \\ \hline
        RF-5 & El sistema permite al usuario acceder a sus contraseñas únicamente a través de la contraseña maestra. \\ \hline
        RF-6 & El sistema requiere 2FA para iniciar sesión desde un dispositivo nuevo, distinto a los utilizados con anterioridad. \\ \hline
        RF-7 & El usuario se registra en el sistema mediante un correo electrónico y una contraseña maestra.\\ \hline
        RF-7.1 & El registro de sesión se deberá confirmar, para verificar la identidad, mediante un correo al usuario registrado. \\ \hline
        RF-8 & El sistema permite la actualización de las contraseñas. \\ \hline
        RF-9 & Se accede al sistema mediante una aplicación móvil. \\ \hline
        RF-10 & Se accede al sistema mediante una interfaz web. \\ \hline
        RF-10 & El sistema ofrece un \textit{plug-in} para navegador web. \\ \hline
    \end{tabular}
\end{table}

\subsection{Diseño del sistema}

\section{Memoria del proyecto}

\subsection{Inicio del proyecto}

\subsection{Ejecución y control del proyecto}

\subsection{Cierre del proyecto}

\section{Conclusiones}

\section*{Glosario}
\addcontentsline{toc}{section}{Glosario}

\section*{Anexo I. Actas de todas las reuniones realizadas}
\addcontentsline{toc}{section}{Anexo I. Actas de todas las reuniones realizadas}

\section{Conclusiones}

\section{Bibliografía}

\end{document}

