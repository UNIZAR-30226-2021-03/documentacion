\documentclass{article}
\usepackage[utf8]{inputenc}
\usepackage[spanish]{babel}
\usepackage[hidelinks]{hyperref}
\usepackage{setspace}
\usepackage{graphicx}
\usepackage{float}
\usepackage[hidelinks]{hyperref}

\usepackage[compact]{titlesec}
\usepackage{geometry}
\geometry{
    a4paper,
    total={170mm,257mm},
    left=30mm, % entre 25-35
    right=30mm, % entre 25-35
    top=30mm, % no más de 30
    bottom=30mm, % no más de 30
}

\usepackage{fancyhdr}
\pagestyle{fancy}
\fancyhf{}
\lhead{}
\rhead{Bárbaros Software S.A.}
\rfoot{\thepage}
\lfoot{Plan de gestión, análisis, diseño y memorial del proyecto KeyPaX}
\renewcommand{\footrulewidth}{0.4pt}

\setlength{\parindent}{2em}
\setlength{\parskip}{1em}

\begin{document}

\begin{titlepage}
    \centering
    \vspace*{\fill}

    \vspace*{0.5cm}

    \Huge
    \textbf{Plan de gestión, análisis, diseño y memorial del proyecto KeyPaX}

    \vspace*{0.5cm}

    \huge
    Bárbaros Software S.A. (Grupo 3)

    \includegraphics[height=5cm]{../images/logo.jpg}

    \begin{Large}
        Carlos Bellvis, 755452\\
        Jorge Bernal, 775695\\
        Jorge Borque, 777959\\
        Arturo Calvera, 776303\\
        Andoni Salcedo, 785649\\
        Javier Vela, 775593\\
    \end{Large}

    \vspace*{1cm}

    \large
    \href{https://github.com/UNIZAR-30226-2021-03}{Organización GitHub: https://github.com/UNIZAR-30226-2021-03}

    \vspace*{\fill}

\end{titlepage}

\pagebreak

\tableofcontents

\pagebreak

\section{Introducción}

\section{Organización del proyecto}

El equipo del proyecto está formado por los 6 integrantes del grupo. Para dividir el trabajo y las responsabilidades se han formado 2 grupos iniciales, acorde con las capacidades y conocimientos individuales. Además, se ha designado como \textbf{director del proyecto} a \textbf{Arturo Calvera}, cuya responsabilidad abarca la gestión de los equipos de trabajo y sus responsabilidades.

\begin{itemize}
    \item \textbf{Equipo \textit{Backend}}:
    \begin{itemize}
        \item Responsable: Arturo Calvera.
        \item Función: Desarrollo del \textit{backend} del sistema.
        \item Integrantes: Arturo Calvera, Andoni Salcedo.
    \end{itemize}
    \item \textbf{Equipo \textit{Frontend}}:
    \begin{itemize}
        \item Responsable: Javier Vela.
        \item Sub-equipos: El equipo dedicado a la vista del sistema se divide para el desarrollo de las dos interfaces.
        \begin{itemize}
            \item \textbf{Equipo Web}:
            \begin{itemize}
                \item Función: Desarrollo del \textit{frontend} web del sistema.
                \item Integrantes: Jorge Bernal, Javier Vela.
            \end{itemize}
        \item \textbf{Equipo Android}:
            \begin{itemize}
                \item Función: Desarrollo de la aplicación cliente del sistema para dispositivos Android.
                \item Integrantes: Carlos Bellvis, Jorge Borque.
            \end{itemize}
        \end{itemize}
    \end{itemize}
    \item Se contempla la posibilidad de crear nuevos equipos para las necesidades que surjan durante el proyecto (\textit{e.g.} despliegue, pruebas) o la reconfiguración de los actuales para adecuarse a la situación.
    %%% IGUAL ESTO NO MENCIONARLO Y ACTUALIZARLO CUANDO SE CREEN NUEVOS EQUIPOS
\end{itemize}

\section{Plan de gestión del proyecto}

\subsection{Procesos}

\subsubsection{Procesos de inicio}

Para asignar los recursos que vamos a utilizar, hemos puesto en común los que conocíamos cada componente del grupo, y a partir de ahí, cuál es el favorito de la mayoría del grupo y en el que nos encontramos más cómodos trabajando.

Los servidores en cloud que más nos llamaron la atención son Amazon Web Services y MIcrosoft Azure. Después de informarnos sobre ambas y compararlas, decidimos crearnos una cuenta en AWS. De primeras, los hemos visto como las 2 plataformas con más impacto a nivel global y que más nos pueden servir para nuestro futuro, a partir de aquí, al no tener ninguno una experiencia previa en ellas, hemos cogido AWS por interés general de todos y por las posibilidades de trabajo que nos ofrece gratuitamente.
Respecto a la base de datos, hemos decidido usar MongoDB ya que ha sido utilizada por la mayoría del grupo en asignaturas previas, y por su compatibilidad con AWS. Para su utilización, nos hemos registrado en Mongo Atlas.

Sobre la formación inicial, hemos tenido un debate amplio a cerca de si es mejor que todos conozcamos todas las tecnologías que vamos a utilizar, o si después de haber organizado el trabajo y el rol de cada uno, se especialice y enfoque su formación en la tecnología que va a utilizar en su caso. Entonces, hemos decidido que todos emplearán tiempo antes de la fecha designada para empezar a realizar el despliegue, en autoformación con tutoriales que vamos poniendo en común sobre las APIs donde vamos a desarrollar nuestro proyecto. Y luego, una vez ya decidido el trabajo de cada uno, enfoque su formación en su tecnología correspondiente, para poder desarrollar sin problemas su tarea y poder ir todo el grupo a un ritmo semejante.


\subsubsection{Procesos de ejecución y control}

\subsubsection{Procesos técnicos}
Para implementar las vistas del sistema se utilizará por una parte Java (Android SDK) para construir el frontend para móviles Android y por otra JavaScript (React.JS) que corresponderá con los distintos navegadores web.
En cuanto al backend, se desarrollará utilizando Node.JS y la framework Express. La base de datos utilizará el SGBD MongoDB.
El despliegue del sistema se realizará en un entorno de contenedores Docker.
Finalmente se ejecutarán unos scripts que llevarán a cabo las pruebas necesarias de las distintas funcionalidades para la comprobación del funcionamiento correcto.
\subsection{Planes}

\subsubsection{Plan de gestión de configuraciones}
A continuación se detallan los planes establecidos para la gestión continua de las configuraciones del proyecto.

Para asegurar la correcta comprensión del código y la navegabilidad del mismo se establece una convención de nombrado de archivos, una estructura de directorios y guías de estilo a seguir a los largo de todos los módulos del proyecto.

\underline{→Nombrado de archivos:}

Todos los archivos del proyecto quedarán nombrados con el siguiente formato:

$<A><B><C><D>$

%cAMBIAR FORMATO
A = Nombre identificativo principal del archivo, el más identificativo.
B = Conjunto de nombres auxiliares opcionales para mejorar la identificación.
C= “Subextensión” opcional para marcar el directorio padre y a su vez tipo de función.
D = Extensión del archivo.

En estos campos solo se permiten cadenas de texto que cumplan la siguiente ER para evitar el uso de caracteres especiales: [a-zA-z]+[0-9]*.
Se intentará además el uso en la medida de lo posible de nombres anglosajones.
Todos los nombres comenzarán por una letra mayúscula. (Campos A y B).

Ejemplos: 

Users.controller.js : Nombre de un archivo en subdirectorio controllers.

FeedUsers.css :  Nombre de un archivo css para componente de feed de usuarios.
\pagebreak

\underline{→ Estructura de directorios:}

Todos los módulos seguirán una estructura de directorios de agrupamiento por tipo de fichero. Es decir, los ficheros quedarán agrupados bajo un directorio padre que indique su uso dentro del módulo o funcionalidad.

Ejemplo de estructura en back-end:

src
|	app.js	\#App entry point
|-------- config 	\#Environment variables and configuration related stuff
|----------> Db.config.js
|-------- models \#Database models
|----------> Users.model.js
|----------> Admins.model.js
| ...


\underline{→ Guías de estilo:}

%REVISAR!!!!

-Estándares de código para desarrollo android: 
AOSP Java Code Style for Contributors.
-Estándares de código para desarrollo en React:
React design principles

Para el control del versionado y actualización del código se utilizarán distintos repositorios de github para los módulos del sistema, es decir, un repositorio de front-end web, de app móvil, de back-end y de documentación del proyecto. A continuación se enumeran los procedimientos a seguir para el uso de estos repositorios.

Dichos repositorios serán privados sólo accesibles por el equipo de desarrollo.
Se asociará a cada repositorio un conjunto de de “GitHub Actions” para administrar la compilación y puesta en marcha del código.
Los commits a estos repositorios irán acompañados de un nombre descriptivo de la tarea asociada.
Los commits podrán hacerse en cualquier momento siempre que se haya probado el código previamente de manera local y ateniéndose al estado de las máquinas de despliegue.
Se seguirá una filosofía de entrega continua y despliegue continuo.
Semanalmente se revisarán los commits realizados para evaluar el progreso de las tareas.

%%INCIDENCIAS?????????????¿?¿¿?¿?¿??¿?¿?¿¿?¿¿?¿?¿?¿?¿¿?¿?¿?¿

\subsubsection{Plan de construcción y despliegue del software}

El sistema se fundamenta en el desarrollo de cuatro subsistemas que trabajan aislados de los otros, siendo su comunicación entre los módulos expuesta a través de interfaces que los relacionan, de este modo tanto las pruebas, compilación y dependencias son independientes al resto de subsistemas. 

Dos de los cuatro subsistemas, el frontend web y el backend, estarán empotrados en contenedores docker y desplegados en la nube utilizando los servicios de \textit{Amazon Web Services}, se utilizarán scripts de \textit{GitHub actions} para automatizar la compilación y testing y despliegue en AWS de los mismos, utilizando en el caso de entorno web librerías de testing como Jest y Postman en el caso de la \textit{API REST}. 

Para el subsistema que concierne a la capa de datos, se despliega en un cluster de MongoDB situado en Bélgica siendo este proporcionado por el equipo de \textit{MongoDB Atlas}, donde se van a desarrollar una serie de diversos triggers que lleven un control de la consistencia de datos tanto tras el uso operaciones además se llevará un control periodico para evitar la replicación y la detección de anomalías a través del uso de scripts.

La interfaz móvil llevará por su cuenta la compilación y testing para cada versión funcional de la aplicación, se integrarán test de unidad, funcionalidad y sobrecarga del sistema.

El punto fuerte de utilizar el despliegue basado en contenedores es que el control de dependencias es indiferente al sistema operativo y al entorno de desarrollo de los programadores, de esta forma cada integrante del equipo podrá configurar y personalizar su entorno por cuenta propia. En el fronted móvil al ser independiente se fija una serie de requisitos para el equipo que trabaja en esta parte, se usará la versión Android 6.1 (\textit{Marshmallow}) y se usará Java como lenguaje de programación, el control de dependencias es llevado por el propio Gradel de Android.

La configuración base de los subsistema será la siguiente, las interfaces del frontend web y el backend están expuestas en el puerto 80, el modelo de capa de datos proporciona una uri externa con la que acceder a la base de datos. las variables de usuarios y contraseñas serán almacenadas como variables de entorno para evitar ponerlas como texto plano en el código.

\subsubsection{Plan de aseguramiento de calidad}

\subsubsection{Calendario del proyecto y división del trabajo}

\pagebreak

\section{Análisis y diseño del sistema}

\subsection{Análisis de requisitos}

\begin{table}[H]
    \centering
    \begin{tabular}{| c | p{30em} |}
    \hline
        Código &  Descripción  \\ \hline
        RF-1 & El sistema permite almacenar contraseñas. \\ \hline
        RF-1.1 & El sistema permite almacenar pares que constan de nombre de usuario y contraseña.  \\ \hline
        RF-1.2 & El sistema permite asociar a las contraseñas una URL del sitio web al que corresponden. \\ \hline
        RF-1.3 & El sistema registra la fecha de creación y actualización de la contraseña. \\ \hline
        RF-1.4 & El sistema permite almacenar una descripción de texto asociada a la contraseña. \\ \hline
        RF-1.5 & El sistema permite almacenar ficheros de imagen (jpeg, png, ...) y ficheros PDF, asociados a la contraseña. \\ \hline
        RF-2 & El sistema permite organizar las contraseñas por categorías, fecha de creación y actualización. \\ \hline
        RF-3 & El sistema permite realizar una búsqueda entre las contraseñas por categoría, nombre de usuario.  \\ \hline
        RF-4 & El sistema permite generación de contraseñas pseudoaleatorias. \\ \hline
        RF-4.1 & El sistema permite seleccionar la longitud de la contraseña a generar.\\ \hline
        RF-4.2 & El sistema permite seleccionar el conjunto de caracteres que compone la contraseña a generar.\\ \hline
        RF-4.3 & El sistema mostrará el grado de robustez de la contraseña al ser generada. \\ \hline
        RF-5 & El sistema permite al usuario acceder a sus contraseñas únicamente a través de la contraseña maestra. \\ \hline
        RF-6 & El sistema requiere 2FA para iniciar sesión desde un dispositivo nuevo, distinto a los utilizados con anterioridad. \\ \hline
        RF-7 & El usuario se registra en el sistema mediante un correo electrónico y una contraseña maestra.\\ \hline
        RF-7.1 & El registro de sesión se deberá confirmar, para verificar la identidad, mediante un correo al usuario registrado. \\ \hline
        RF-8 & El sistema permite la actualización de las contraseñas. \\ \hline
        RF-9 & Se accede al sistema mediante una aplicación móvil. \\ \hline
        RF-10 & Se accede al sistema mediante una interfaz web. \\ \hline
        RF-10 & El sistema ofrece un \textit{plug-in} para navegador web. \\ \hline
    \end{tabular}
\end{table}

\subsection{Diseño del sistema}

\section{Memoria del proyecto}

\subsection{Inicio del proyecto}

\subsection{Ejecución y control del proyecto}

\subsection{Cierre del proyecto}

\section{Conclusiones}

\section*{Glosario}
\addcontentsline{toc}{section}{Glosario}

\section*{Anexo I. Actas de todas las reuniones realizadas}
\addcontentsline{toc}{section}{Anexo I. Actas de todas las reuniones realizadas}

\section{Conclusiones}

\section{Bibliografía}

\end{document}
