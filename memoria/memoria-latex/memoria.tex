\documentclass{article}
\usepackage[utf8]{inputenc}
\usepackage[spanish]{babel}
\usepackage[hidelinks]{hyperref}
\usepackage{setspace}
\usepackage{graphicx}
\usepackage[hidelinks]{hyperref}

\usepackage[compact]{titlesec}
\usepackage{geometry}
\geometry{
    a4paper,
    total={170mm,257mm},
    left=30mm, % entre 25-35
    right=30mm, % entre 25-35
    top=30mm, % no más de 30
    bottom=30mm, % no más de 30
}

\usepackage{fancyhdr}
\pagestyle{fancy}
\fancyhf{}
\lhead{}
\rhead{Bárbaros Software S.A.}
\rfoot{\thepage}
\lfoot{Plan de gestión, análisis, diseño y memorial del proyecto KeyPaX}
\renewcommand{\footrulewidth}{0.4pt}

\setlength{\parindent}{2em}
\setlength{\parskip}{1em}

\begin{document}

\begin{titlepage}
    \centering
    \vspace*{\fill}

    \vspace*{0.5cm}

    \Huge
    \textbf{Plan de gestión, análisis, diseño y memorial del proyecto KeyPaX}

    \vspace*{0.5cm}

    \huge
    Bárbaros Software S.A. (Grupo 3)

    \includegraphics[height=5cm]{../images/logo.jpg}

    \begin{Large}
        Carlos Bellvis, 755452\\
        Jorge Bernal, 775695\\
        Jorge Borque, 777959\\
        Arturo Calvera, 776303\\
        Andoni Salcedo, 785649\\
        Javier Vela, 775593\\
    \end{Large}

    \vspace*{1cm}

    \large
    \href{https://github.com/UNIZAR-30226-2021-03}{Organización GitHub: https://github.com/UNIZAR-30226-2021-03}

    \vspace*{\fill}

\end{titlepage}

\pagebreak

\tableofcontents

\pagebreak

\section{Introducción}

\section{Organización del proyecto}

El equipo del proyecto está formado por los 6 integrantes del grupo. Para dividir el trabajo y las responsabilidades se han formado 2 grupos iniciales, acorde con las capacidades y conocimientos individuales. Además, se ha designado como \textbf{director del proyecto} a \textbf{Arturo Calvera}, cuya responsabilidad abarca la gestión de los equipos de trabajo y sus responsabilidades.

\begin{itemize}
    \item \textbf{Equipo \textit{Backend}}:
    \begin{itemize}
        \item Responsable: Arturo Calvera.
        \item Función: Desarrollo del \textit{backend} del sistema.
        \item Integrantes: Arturo Calvera, Andoni Salcedo.
    \end{itemize}
    \item \textbf{Equipo \textit{Frontend}}:
    \begin{itemize}
        \item Responsable: Javier Vela.
        \item Sub-equipos: El equipo dedicado a la vista del sistema se divide para el desarrollo de las dos interfaces.
        \begin{itemize}
            \item \textbf{Equipo Web}:
            \begin{itemize}
                \item Función: Desarrollo del \textit{frontend} web del sistema.
                \item Integrantes: Jorge Bernal, Javier Vela.
            \end{itemize}
        \item \textbf{Equipo Android}:
            \begin{itemize}
                \item Función: Desarrollo de la aplicación cliente del sistema para dispositivos Android.
                \item Integrantes: Carlos Bellvis, Jorge Borque.
            \end{itemize}
        \end{itemize}
    \end{itemize}
    \item Se contempla la posibilidad de crear nuevos equipos para las necesidades que surjan durante el proyecto (\textit{e.g.} despliegue, pruebas) o la reconfiguración de los actuales para adecuarse a la situación.
    %%% IGUAL ESTO NO MENCIONARLO Y ACTUALIZARLO CUANDO SE CREEN NUEVOS EQUIPOS
\end{itemize}

\section{Plan de gestión del proyecto}

\subsection{Procesos}

\subsubsection{Procesos de inicio}

\subsubsection{Procesos de ejecución y control}

\subsubsection{Procesos técnicos}
Para implementar las vistas del sistema se utilizará por una parte Java (Android SDK) para construir el frontend para móviles Android y por otra JavaScript (React.JS) que corresponderá con los distintos navegadores web.
En cuanto al backend, se desarrollará utilizando Node.JS y la framework Express. La base de datos utilizará el SGBD MongoDB.
El despliegue del sistema se realizará en un entorno de contenedores Docker.
Finalmente se ejecutarán unos scripts que llevarán a cabo las pruebas necesarias de las distintas funcionalidades para la comprobación del funcionamiento correcto.
\subsection{Planes}

\subsubsection{Plan de gestión de configuraciones}

\subsubsection{Plan de construcción y despliegue del software}

El sistema se fundamenta en el desarrollo de cuatro subsistemas que trabajan aislados de los otros, siendo su comunicación entre los módulos expuesta a través de interfaces que los relacionan, de este modo tanto las pruebas, compilación y dependencias son independientes al resto de subsistemas. 

Dos de los cuatro subsistemas, el frontend web y el backend, estarán empotrados en contenedores docker y desplegados en la nube utilizando los servicios de \textit{Amazon Web Services}, se utilizarán scripts de \textit{GitHub actions} para automatizar la compilación y testing y despliegue en AWS de los mismos, utilizando en el caso de entorno web librerías de testing como Jest y Postman en el caso de la \textit{API REST}. 

Para el subsistema que concierne a la capa de datos, se despliega en un cluster de MongoDB situado en Bélgica siendo este proporcionado por el equipo de \textit{MongoDB Atlas}, donde se van a desarrollar una serie de diversos triggers que lleven un control de la consistencia de datos tanto tras el uso operaciones además se llevará un control periodico para evitar la replicación y la detección de anomalías a través del uso de scripts.

La interfaz móvil llevará por su cuenta la compilación y testing para cada versión funcional de la aplicación, se integrarán test de unidad, funcionalidad y sobrecarga del sistema.

El punto fuerte de utilizar el despliegue basado en contenedores es que el control de dependencias es indiferente al sistema operativo y al entorno de desarrollo de los programadores, de esta forma cada integrante del equipo podrá configurar y personalizar su entorno por cuenta propia. En el fronted móvil al ser independiente se fija una serie de requisitos para el equipo que trabaja en esta parte, se usará la versión Android 6.1 (\textit{Marshmallow}) y se usará Java como lenguaje de programación, el control de dependencias es llevado por el propio Gradel de Android.

\subsubsection{Plan de aseguramiento de calidad}

\subsubsection{Calendario del proyecto y división del trabajo}

\section{Análisis y diseño del sistema}

\subsection{Análisis de requisitos}

\subsection{Diseño del sistema}

\section{Memoria del proyecto}

\subsection{Inicio del proyecto}

\subsection{Ejecución y control del proyecto}

\subsection{Cierre del proyecto}

\section{Conclusiones}

\section*{Glosario}
\addcontentsline{toc}{section}{Glosario}

\section*{Anexo I. Actas de todas las reuniones realizadas}
\addcontentsline{toc}{section}{Anexo I. Actas de todas las reuniones realizadas}



\section{Conclusiones}

\section{Bibliografía}

\end{document}
